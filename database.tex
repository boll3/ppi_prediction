\documentclass{article}
\usepackage{graphicx}
\usepackage{color}


\newcommand{\db}{\texttt{PPromptDB}}
\newcommand{\url}[1]{\texttt{http://\small #1}}
\newcommand{\server}{ \url{proteinformatics.org/\db}}



\title{\db: a comprehensive and curated database of protein-protein interactions}

\author{ Sebastian Canzler$^{1,3}$, Ren\'{e} Staritzbichler$^{2,3}$}
\date{}



\begin{document}

\maketitle


\begin{abstract}
  Protein-protein interactions are among the key-drivers of biology and thus important targets for medical regulation. 
  Here, we present a comprehensive and curated database of both positive and negative protein-protein contacts.
  \db\ is available for download under \server.
  
  
\end{abstract}



\section*{Introduction}
  Proteins are the 'machines' of the cell, taking on manifold roles as receptors, enzymes, transporters or even molecular factories.
  Most biological processes involve the interaction between proteins, both in healthy systems as well as in disease.
  Therefore, most drugs target proteins and often impact protein-protein interactions - whether intended as desired effect or unexpected as side-effects.

  
  Computation of the binding behaviour of proteins can complement and guide experimental studies as well as the development of protein-based therapeutics.
  However, calculating protein binding in atomic detail is highly time consuming and requires  knowledge about the structures.
  While proteins are complex three-dimensional molecules, their structure, function and also binding behaviour is encoded in their amino acid sequence.
  In turn, the interaction between proteins can be understood to some degree from their sequence alone.

  Hence, protein-protein interactions are a field, where the application of

  No real high quality predictor available yet.
  We provide the \db\ 


  
  For the training of a high quality learning algorithm a comprehensive dataset needs to be collected.
  We exploited all available sources.
  Positive data was gathered from the
  Database of Interacting Proteins (DIP)
  \cite{Salwinski:2004},
  Human Protein Reference Database (HPRD)
  \cite{Keshava_Prasad:2009},
  Protein  Data Bank (PDB) \cite{Berman:2000}, and the University of Kansas Proteomics Service (KUPS) \cite{Chen:2011}.
  Negative data was obtained from the Negatome Database \cite{Blohm:2014}  and the KUPS server.
  After collecting such a broad database, thorough curation was imperative to improve data quality.
  We performed several iterations of filtering and cleanup, both automated in terms of computational tools, as well as through manual inspection.
  
 \section*{Methods}

In order to create comprehensive training and testing data we tried to
collect as many trustworthy PPI annotations as possible. We included
data from various sources such as \texttt{Database of Human
  Interacting Proteins}\footnote{http://dip.doe-mbi.ucla.edu/} (DIP)
\cite{Salwinski:2004}, \texttt{Human Protein Reference
  Database}\footnote{http://www.hprd.org/} (HPRD)
\cite{Keshava_Prasad:2009}, \texttt{Protein
  Database}\footnote{www.rcsb.org} (PDB) \cite{Berman:2000}, and the
\nega\footnote{http://mips.helmholtz-muenchen.de/proj/ppi/negatome/}
\cite{Blohm:2014}. We also included annotations retrieved from the
\kups \footnote{http://www.ittc.ku.edu/chenlab/kups/}   server
\cite{Chen:2011} which mainly incorporates PPIs from
\texttt{MINT}\footnote{https://mint.bio.uniroma2.it/}
\cite{Licata:2012} and
\texttt{IntAct}\footnote{https://www.ebi.ac.uk/intact/}
\cite{Orchard:2014}. In addition to the negative annotations collected
from the \nega, the \kups\ server generated negative data-points based
on the the following criteria: (1) proteins should be functional
dissimilar, (2) proteins should be localized in different cellular
compartments, and (3) proteins are part of non-interacting domains. 

After intense manual curation and mapping of different names describing
the exact same protein, we derived a total amount of 31,867 distinct
human proteins. For this set, 73,681 positive PPIs were collected
from the datasources previously mentioned. By means of \texttt{CD-hit}
\cite{Li:2006, Fu:2012}, we reduced this set to 41,844 positive
protein-protein pairs with at most 50\% sequence identity. For
negative pairs, we collected over 1.5 million unique protein-protein
pairs, from which we randomly selected a number of PPIs, that was
matching the size of the positive dataset.



 \section*{Results}

 
 \section*{Conclusions}


\end{document}


