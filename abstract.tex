\documentclass{article}
\usepackage{graphicx}
\usepackage{color}
\usepackage{amssymb}
\usepackage{amsmath}
\usepackage{multirow}
\usepackage{multicol}
\usepackage{array}
\usepackage{rotating,capt-of}
\usepackage[small]{caption}
\usepackage{booktabs}
%\usepackage{float} % for placing figures where i want
\usepackage{afterpage}
\usepackage{epsfig, a4wide}
\usepackage{titling} % shifting title
%\usepackage[margin=1in]{geometry}

\usepackage{tikz,graphicx}
\usetikzlibrary{shapes.geometric, arrows}
\usetikzlibrary{shapes.misc, positioning}
\usetikzlibrary{backgrounds}

\definecolor{lavander}{cmyk}{0,0.48,0,0}
\definecolor{violet}{cmyk}{0.79,0.88,0,0}
\definecolor{burntorange}{cmyk}{0,0.52,1,0}

\def\lav{lavander!90}
\def\oran{orange!30}
\def\dgre{dgreen!30}
\def\vio{violet!30}

\tikzstyle{neighbors}=[draw,circle,violet,bottom color=\lav,
                  top color= white, font=\scriptsize,text=violet,minimum width=10pt]
\tikzstyle{queries}=[draw,circle,burntorange, left color=\oran,
                       text=violet,minimum width=30pt]

\tikzstyle{io} = [trapezium, trapezium left angle=70, trapezium right angle=110, minimum width=3cm, minimum height=1cm, text centered, draw=black, color=burntorange,left color=\oran,text=black,font=\Large]

\tikzstyle{res} = [trapezium, trapezium left angle=70, trapezium right angle=110, minimum width=3cm, minimum height=1cm, text centered, draw=black, color=violet,bottom color=\lav, top color=white,text=black,font=\Large]

\tikzstyle{process} = [rectangle, minimum width=3cm, minimum height=1cm, text centered, draw=violet, bottom color=\vio, top color=white,font=\Large]

\tikzstyle{propro} = [rectangle, minimum width=3cm, minimum height=1cm, text centered, draw=violet, bottom color=\dgre, top color=white,font=\Large]

\tikzstyle{grey} = [ rectangle, rounded corners=10pt, bottom color=black!10, top color=black!2,font=\Large]

\tikzstyle{aro} = [->, thick,  shorten >=2pt, shorten <=2pt]



\newcommand{\myparagraph}[1]{
  \paragraph*{\normalfont\itshape #1}\hspace{5pt}}

% strange snos
\definecolor{purple}{RGB}{180,90,200}
\definecolor{dgreen}{RGB}{0,160,0}
\definecolor{turquoise}{RGB}{0,180,140}
\renewcommand\dblfloatpagefraction{0.03}
\renewcommand\topfraction{.95}
\renewcommand\bottomfraction{.95}
\renewcommand\textfraction{.05}
\renewcommand\floatpagefraction{.95}
\renewcommand\dbltopfraction{.95}
\renewcommand\dblfloatpagefraction{.95}
\newcommand{\TODO}[1] {\begingroup\color{red}#1\endgroup}
\newcommand{\SC}[1] {\begingroup\color{purple}#1\endgroup}
\newcommand{\ACC}[1]{\emph{\textbf{#1}}}
\newcommand{\s}[1]{\begin{tiny}#1\end{tiny}}
\newcommand{\url}[1]{\texttt{http://\small #1}}
\newcommand{\maxentscan}{\texttt{MaxEntScan}}
\newcommand{\NEW}[1]{\begingroup\color{black}#1\endgroup}

%% programs
\newcommand{\spps}{\texttt{SPPS}}
\newcommand{\tri}{\texttt{TRI\_tool}}
\newcommand{\lr}{\texttt{LR\_PPI}}

%% databases
\newcommand{\ncbi}{\texttt{NCBI}}
\newcommand{\nega}{\texttt{Negatome Database}}
\newcommand{\kups}{\texttt{KUPS}}

%\newcommand{\tool}{\texttt{rfPRO}}
%\newcommand{\tool}{\texttt{jackProt}}
\newcommand{\tool}{\texttt{ProteinPrompt}}
\newcommand{\website}{\url{proteinformatics.org/\tool}}

\newcommand{\Hsa}{\emph{Homo sapiens}}
\newcommand{\hsa}{\emph{H.sapiens}}

%\journal{Nature}

\setlength{\droptitle}{-5em}

%\bibliographystyle{naturemag}
\title{\tool: fast and accurate prediction of protein-protein-interactions}

\author{ Sebastian Canzler$^{1,3}$, Ren\'{e} Staritzbichler$^{2,3}$}
\date{}



\begin{document}


\maketitle


\begin{abstract}

  Protein-protein interactions are among the key-drivers of biology and thus important targets for medical regulation. 
  Here, we present \tool, a precise machine learning method for the calculation of protein-protein interaction-networks.
  To make it accessible for a large community, \tool is available online as a webserver:

  \noindent \website.
  The server allows scanning the human proteasome, and several other databases, for potential binding partners of the query in unprecedent accurracy.
  A scan through the entire human proteasome is performed within minutes.
  

  Learning methods depend crucially on both size and quality of datasets used for training and testing.
  Starting point of ProteinPrompt was therefore the collection of a comprehensive
  dataset from basically all available sources.
  In turn, a very thorough filtering was imperative.
  The actual keystep is the transformation of the original sequence data into a representation that enables the learning
  algorithm to understand the underlying patterns that lead to binding versus non-binding.
  After comparing various choices of learning algorithms and descriptors, the final version of \tool\ exploits the random forest algorithm using auto-correlation of seven amino-acid scales.
  As a result of exhaustive optimizations, \tool is reaching an area under curve of 0.95, which is far above any other publicly available tool.

\end{abstract}



\noindent\textbf{Affiliations:}\\
  { 1) Bioinformatics Group, Department of Computer Science,
  University of Leipzig,
  H{\"a}rtelstra{\ss}e 16-18, 04107 Leipzig, Germany.\\
2) ProteinFormatics Group, Institute of Medical Physics and Biophysics, University of Leipzig,
  H{\"a}rtelstra{\ss}e 16-18, 04107 Leipzig, Germany.\\
3)
Immuthera GmbH, L{\"o}{\ss}niger Stra{\ss}e 16, 04275 Leipzig, Germany.}




\end{document}


